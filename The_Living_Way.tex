\documentclass[10pt, openright]{book}

% --- KDP 5x8 PAPERBACK CONFIGURATION ---
% 5x8 inch trim size.
% 'inner' margin is larger (0.75in) to account for the spine binding (gutter).
% 'outer' margin is smaller (0.5in) to maximize reading space.
\usepackage[paperwidth=5in, paperheight=8in, top=0.75in, bottom=0.75in, outer=0.5in, inner=0.75in]{geometry}

% --- TYPOGRAPHY ---
\usepackage[T1]{fontenc}
\usepackage[utf8]{inputenc}
\usepackage[english]{babel}
\usepackage{ebgaramond} % The Classic Serif Font (Time-honored and legible)
\usepackage{microtype}  % Essential for professional typesetting (smooths edges)

% --- STYLING PACKAGES ---
\usepackage{titlesec}
\usepackage{fancyhdr}
\usepackage{parskip}
\usepackage{amssymb}
\usepackage{emptypage} % Prevents page numbers/headers on blank pages

% --- 1. CHAPTER STYLING (The "Books") ---
\titleformat{\chapter}[display]
  {\normalfont\centering} % Global style for chapter
  {\vspace{-1cm}\itshape\large Book \thechapter} % Label "Book I"
  {10pt} % Space between label and title
  {\Huge\bfseries} % Title style
\titlespacing*{\chapter}{0pt}{0pt}{40pt}

% --- 2. SAYING STYLING (The "Statuary" Look) ---
% We use [display] to stack the number above the title
\titleformat{\section}[display]
  {\normalfont\centering} % Center everything
  {\vspace{1em}\Huge\bfseries\thesection} % THE NUMBER: Huge and Bold
  {0.3em} % Small gap between number and title
  {\large\scshape} % THE TITLE: Small Caps, slightly smaller than number
  
% Adjust spacing: {left}{top}{bottom}
\titlespacing*{\section}{0pt}{1.0ex plus 1ex minus .2ex}{1.5ex plus .2ex}

% --- 3. PAGE BREAK LOGIC ---
% CRITICAL: This command ensures a saying is never split across pages.
% If it doesn't fit, it moves to the next page entirely.
\newcommand{\sectionbreak}{\filbreak}

% --- 4. ORNAMENTS ---
\newcommand{\ornament}{
    \vspace{1.5cm}
    \begin{center}
        \large $\cdot$ $\odot$ $\cdot$
    \end{center}
    \vspace{1.5cm}
}

% --- 5. HEADER/FOOTER ---
\pagestyle{fancy}
\fancyhf{}
\fancyhead[CO]{\scshape\small The Scroll of the Living Way} 
\fancyhead[CE]{\scshape\small Yeshua the Living One}
\fancyfoot[C]{\thepage}
\renewcommand{\headrulewidth}{0pt}

\begin{document}

% --- TITLE PAGE ---
\begin{titlepage}
    \centering
    \vspace*{1.5cm}
    {\Huge \textsc{The Scroll of}\\ \textsc{The Living Way}} \\
    \vspace{0.5cm}
    {\large \textit{The 81 Sayings of Yeshua the Living One}} \\
    \vspace{2cm}
    \textbf{A Gnostic Tao for the Children of Light}
    \vfill
    {\small Compiled \& Prepared for the Seeker of Gnosis}
    \vspace{1cm}
\end{titlepage}

\frontmatter
\tableofcontents

\chapter{Preface}
\textit{Yeshua spoke:}

Beloved, \\
you wander in search of what has never been lost. \\
You lift your eyes to heaven, \\
yet the Kingdom is already laid in your breath. \\
You seek the Light in scriptures and temples, \\
yet the Light waits behind your seeing.

If you would know me, \\
enter the silence where your name has not yet formed. \\
There the Father of Light whispers, \\
and the Mother of Wisdom gathers you to Herself. \\
There the two are made one, \\
and the one is shown to be none.

Do not follow me as a man follows another man. \\
Follow the One who moves within your own heart, \\
for I am that One \\
when you remember yourself. \\
And I am hidden from you \\
only when you forget.

These sayings are not commandments \\
but mirrors. \\
Look into them \\
until you see your own face shining there— \\
the face you had \\
before the world divided you.

When you make the inner like the outer, \\
the above like the below, \\
the lion like the child, \\
and the child like the unbegotten, \\
then the bridal chamber opens, \\
and you awaken as the Living One.

Come, \\
remove your old garments. \\
Step naked into the light of your own being. \\
There is nothing to fear. \\
What you take to be your shadow \\
is only the veil you have outgrown.

Let the scroll unfold within you. \\
Let the sayings ripen in your silence. \\
And when the fruit is sweet, \\
eat and become what you truly are.

\vspace{1cm}
\begin{center}
    \textit{May the Inner Sun rise in your heart.}
\end{center}
\ornament

\mainmatter

% =========================================================
% BOOK 1
% =========================================================
\chapter{The Light Before the World}
\begin{center}
    \textit{(Sayings 1--9)} \\
    \vspace{0.5cm}
    \small From the Silence, the Word. \\ From the Word, the Light.
\end{center}
\vspace{0.5cm}

\section{The Way That Is Not Seen}
The Word caught in ink is a husk; \\
the Word lived is a devouring fire. \\
The Kingdom is spread upon the earth, \\
yet few behold it. \\

Those who hunt with the eyes \\
find only shadows. \\
Those who rest in the still heart \\
find the Source \\
from which all seeing springs.

\section{The Two Become One}
The world is born of the cut in two— \\
male and female, light and shadow, life and death. \\
Yet these divisions are garments only. \\

Truly I say to you: \\
when the two are made one, \\
the lion within you lies down as a child, \\
and the child awakens \\
as the One who is not born. \\

Then the bridal chamber opens in your depths.

\section{The Poverty of the Unknowing}
People cling to the loud voices of the world. \\
They trust laws more than their own heart, \\
beliefs more than living knowing. \\
Thus they sit in a treasury, \\
yet starve. \\

One who knows the self \\
no longer hoards teachings in a scroll. \\
Their very being becomes the scroll.

\section{The Father of Light}
There is a Light that gives birth to itself— \\
unborn, unbroken, unbound. \\
From this Light you came; \\
to this Light you return when you awaken. \\

The Father bears no wrath. \\
The Source casts no shadow. \\
Only the garments of the soul \\
weave darkness.

\section{The Powers Question the Soul}
Desire whispers, ``You are mine.'' \\
Ignorance declares, ``You do not know yourself.'' \\
Anger cries, ``Strike, and be struck.'' \\
The body murmurs, ``I am all there is.'' \\

But the awakened soul smiles and replies: \\
``You never touched me. \\
You grasped only my clothing.''

\section{The Hollow Reed}
The false self is a wall of stone— \\
it resists the wind and crumbles. \\
The soul is a hollow reed— \\
it lets the Spirit breathe through it. \\

When you are empty, \\
even the storm becomes a song.

\section{The Unforced Kingdom}
The Kingdom comes by neither effort nor delay. \\
When you stop grasping at life, \\
life reveals itself as eternal. \\

When you cease seeking the Kingdom, \\
you discover you never left it. \\

What you call ``sin'' is forgetting. \\
What you call ``salvation'' is remembering.

\section{The Servant at the Feast}
Be as the servant who seeks the lowest place at the feast. \\
He does not contend, \\
yet the Master calls him forward. \\

So too the one who knows the Father \\
takes the lowest seat \\
and finds there the highest. \\

Humility is not making the self small; \\
it is seeing that the self is a shadow.

\section{The Teacher Who Does Not Teach}
I do not command belief. \\
I uncover what is hidden, \\
that you may see with your own eyes. \\

A master gives answers; \\
I give you questions— \\
that you may become what I am.

\ornament

% =========================================================
% BOOK 2
% =========================================================
\chapter{The Kingdom Within}
\begin{center}
    	extit{(Sayings 10--18)} 
    \vspace{0.5cm}
    \small The door is shut, yet the house is vast. \ The lamp is small, yet it lights the world.
\end{center}
\vspace{0.5cm}

\section{The Single Eye}
If the eye of the body is divided, 
it sees only confusion. 
If the eye of the heart is single, 
it fills the whole body with light. 

The Nous stands between soul and spirit— 
a lamp lit from both sides. 
Guard this lamp, 
and the night cannot touch you.

\section{The Inner and the Outer}
The cup is precious for its emptiness. 
So too the self is precious 
when it is emptied of self. 

Inside and outside are mirrors. 
When they reflect without distortion, 
the All is revealed.

\section{The Lion and the Child}
The world teaches you to be a lion— 
to conquer, to claim, to devour. 
But I tell you: unless the lion becomes a child, 
you will not know the Living Source. 

The child is open, unguarded, whole. 
There is no image of self to defend. 
Therefore the child steps into the Kingdom easily.

\section{The Unbinding}
What binds you is not outside you. 
What frees you is not outside you. 
Prison and key 
are fashioned from the same ignorance. 

Awaken, and you will laugh— 
for there were never any gates.

\section{The Return to the Unborn}
Before you were a form, you were the Breath. 
Before you were breath, you were the Light. 
Before you were the Light, 
you were hidden in the Source. 

To return is not to go backwards; 
it is to remember the place 
that has never moved.

\section{The Silence That Speaks}
Words are nets thrown into the sea. 
They catch small fish, 
but the Great Fish slips through. 

Do not hunger for many words from me. 
Hunger for the Silence 
in which the Word is born. 
There you and I are one life, 
one breath, 
one Being.

\section{The Great Stillness}
Empty yourself of what you think you are. 
Let the waters of the heart grow still. 
When the mud settles by itself, 
the bottom is revealed. 

So too, when the soul becomes quiet, 
the Light of the Living Source 
shines through without effort.

\section{The Hidden Guide}
The greatest guide is the one who disappears. 
The seeker awakens and cries: 
``Look— 
I have found the Way myself.'' 

Thus I leave no trace, 
and yet my footprints are everywhere.

\section{When the Way Is Forgotten}
When the Way is forgotten, virtue is invented. 
When virtue is lost, rules are created. 
When rules fail, 
people cling to punishment and reward. 

The further you descend, 
the further you move from Life. 
Return to the beginning, 
and the rules fall away like old clothing.

\ornament

% =========================================================
% BOOK 3
% =========================================================
\chapter{The Garment of Silence}
\begin{center}
    	extit{(Sayings 19--27)} 
    \vspace{0.5cm}
    \small Silence is the language of the Father. \ All else is poor translation.
\end{center}
\vspace{0.5cm}

\section{Beyond the Teachings}
Abandon holiness; return to wholeness. 
Abandon righteousness; return to sight. 
Abandon the law; return to the Source. 

What you call ``good'' and ``evil'' 
are shadows cast by your dividing mind. 
The heart that truly sees 
needs no commandments.

\section{Not of This World}
The world shouts; the Way whispers. 
The world demands belief; the Way invites seeing. 
The world praises power; the Way dissolves it. 

Many chase desire; 
few seek liberation. 
Yet the Kingdom belongs 
to those who loosen their grip on everything.

\section{The Face of the Invisible}
The Way is a childlike mystery— 
seen only by those who forget themselves. 

It has no beginning and no end. 
It moves without moving. 
Its face cannot be drawn, 
yet it shines through your own 
when you remember who you are.

\section{The One Who Is Bent}
If you wish to be whole, let yourself be broken. 
If you wish to be full, let yourself be emptied. 
If you wish to be reborn, die to your own image. 

I do not raise the proud. 
I raise the one 
who has laid the self down.

\section{The Wind Speaks Briefly}
Speak only what is true, and your words will be few. 
The wind does not shout, 
yet whole forests bow before it. 

When you speak from the Source, 
your words carry no force, 
yet mountains move.

\section{The Tower That Topples}
The one who raises the self has already fallen. 
The one who displays virtue has already lost it. 
The one who demands honor 
has already betrayed the soul. 

Stand in your true nature, 
and honor flows from you 
without your asking.

\section{The Womb of Silence}
There is a womb older than creation. 
It is the Silence that births the All. 

It is formless, without boundary. 
It moves through you as breath, 
yet it is not the breath. 

Those who know Her 
do not cling to life 
and do not fear death.

\section{The Heavy Root}
The root of the tree is unseen, 
yet it holds the whole tree upright. 
So too the hidden depth of your being 
supports your every step. 

Those who forget their root 
are blown by every wind— 
desire, fear, anger, ignorance. 
Remember your depth, 
and nothing can uproot you.

\section{The Footprints of the Master}
A true master walks without leaving marks, 
speaks without wounding, 
guides without controlling. 

To the blind—a lamp. 
To the lost—a doorway. 
To the weary—rest. 

Seeing the spark in all beings, 
they count none as beyond help.

\ornament

% =========================================================
% BOOK 4
% =========================================================
\chapter{The Power of the Gentle}
\begin{center}
    	extit{(Sayings 28--36)} 
    \vspace{0.5cm}
    \small The soft overcomes the hard. \ The slow overcomes the fast.
\end{center}
\vspace{0.5cm}

\section{The Fertile Soil}
Know the strength of the sower, 
yet keep the softness of the soil— 
the open field that receives the seed. 

Know knowledge, 
yet return to innocence. 
Thus you become the place 
where the two are made one.

\section{The Futility of Control}
Those who try to seize the world will wound it. 
The world is a sacred vessel; 
it cannot be shaped by human hands. 

Force cracks the clay. 
Grasping spills the wine. 
Only the open hand 
receives the fullness of life.

\section{The Way of Non-Violence}
Those who walk in the Way do not harm. 
They do not conquer, 
they do not take up the sword. 

Every blow you strike 
circles back to you. 
Every wound you inflict 
becomes your own garment of pain. 

Power gained through violence rots the soul; 
power born of awakening 
cannot be taken away.

\section{The Weapon of the Heart}
Weapons are the tools of fear; 
the awakened heart has no use for them. 
Victory through harm 
is mourning in disguise. 

The wise win by dissolving conflict, 
not by defeating an enemy— 
seeing no enemy at all.

\section{The Way, the Truth}
People say, ``Show us the path.'' 
I tell them: 	extbf{I am the Way.} 

They say, ``Give us a doctrine.'' 
I tell them: 	extbf{I am the Truth.} 

They say, ``Grant us immortality.'' 
I tell them: 	extbf{I am the Living One.} 

The Way is not a map; 
it is the Single One walking. 
No one comes to the Father of Light 
except by becoming what I am.

\section{Profit and Loss}
To know others is cleverness; 
to know yourself is illumination. 
To conquer others is strength; 
to conquer yourself is freedom. 

What gain is there in winning the ten thousand things 
if you lose the One who sees them?

\section{The Great River}
The Way flows everywhere, 
giving itself to all beings, claiming nothing. 

One who walks in this Way 
acts without self, 
gives without calculation, 
rests without pride— 
and becomes as the Source itself.

\section{The Face of Peace}
Hold fast to the image of the Living One, 
and all beings come to rest in your presence. 

Peace is not the absence of conflict; 
it is the seeing that conflict was a dream. 
The awakened do not persuade— 
they radiate.

\section{The Paradox of Power}
To let something expand, first allow it to contract. 
To let something grow strong, first allow it to weaken. 

So I guide by reversal, 
teaching the soul through its own emptiness.

\ornament

% =========================================================
% BOOK 5
% =========================================================
\chapter{The Union of Opposites}
\begin{center}
    	extit{(Sayings 37--45)} 
    \vspace{0.5cm}
    \small Light and shadow are twins. \ The awakened one sees the Father in both.
\end{center}
\vspace{0.5cm}

\section{The Ease of the Way}
The Way moves without effort. 
When the heart aligns with it, 
desire loosens its grip. 

When desire rises, the soul grows troubled. 
Turn the Light upon the waters 
and they grow clear again.

\section{True Virtue}
High virtue does not display itself; 
it acts from the Source without strain. 
Lesser virtue clings to rules, 
having forgotten its origin. 

When the Kingdom is remembered, 
virtue flowers by itself. 
When it is forgotten, 
people invent commandments.

\section{The Ones Who Remain Whole}
Heaven is Heaven 
because it does not exalt itself. 
Earth is earth 
because it does not resist its nature. 

The soul remains whole 
when it refuses to be divided.

\section{The Return}
The motion of the Way is return— 
not to the past, 
but to the unborn Light within. 

All things rise from the Source, 
and the Source rises from Silence.

\section{The Three Seekers}
The wise hear the teaching and practice deeply. 
The average hear it and practice partially. 
The foolish hear it and laugh. 

Yet the Way dances in their laughter as well, 
for every path, crooked or straight, 
leans toward awakening.

\section{The Birth of the Two}
The Source gives birth to Father and Mother. 
The Two give birth to the Many. 
The Many return to the One. 

The one who sees the One 
in every form and motion 
cannot be shaken.

\section{The Gentle Overcomes}
Nothing is softer than water, 
yet nothing wears away stone more patiently. 

What has no hardness 
enters where nothing hard can enter. 

So the soul, made subtle, 
passes through every barrier.

\section{Treasure and Self}
Fame or life— 
which is more precious? 
Gain or the soul— 
which is more valuable? 

The one who knows their true nature 
cannot be lured away from it.

\section{The Great Fulness}
The cup most full 
often appears most empty. 
The path most straight 
often looks crooked. 
The hand most skilled 
often seems to tremble. 

The world sees only surfaces; 
the Real dwells beneath the seen. 

The awakened befriend paradox, 
and so are not deceived by appearances.

\ornament

% =========================================================
% BOOK 6
% =========================================================
\chapter{The Empty Vessel}
\begin{center}
    	extit{(Sayings 46--54)} 
    \vspace{0.5cm}
    \small Become empty, that you may be filled. \ Become nothing, that you may be All.
\end{center}
\vspace{0.5cm}

\section{The Yoke of Peace}
When the Way is lived, 
the ox knows its master's stall. 
When the Way is forgotten, 
the sword gleams in the street. 

The yoke of the world is heavy with desire. 
My yoke is easy, 
and brings the soul to rest.

\section{Seeing Without Seeking}
Without leaving home, 
you may know the whole world. 
Without using your eyes, 
you may see the Living Source. 

The farther you flee outward, 
the farther you stray within.

\section{The Unlearning}
The scholar gathers more each day. 
The seeker of the Way 
lays down something each day. 

She sheds garments of opinion 
until only the naked Truth remains. 

The one who lets go 
holds the All.

\section{The Heart of the Single One}
The Single One is good to the good, 
and good to those who are not good— 
seeing only the hidden Light. 

Trusting those who trust, 
and trusting those who do not trust— 
for trust rises from within, 
not from them.

\section{Life and Death}
Those who cling to life fear death. 
Those who know Me 
fear neither. 

Death is but the undressing of the bride 
before she enters the chamber.

\section{The Nourishing Way}
The Way gives birth, nourishes, guides, protects, 
and does so without claiming ownership. 

The Single One completes the work, 
forgets the work, 
and the work remains forever.

\section{Return to the Mother}
Know the Mother, and you know the children. 
Know the children, and return to the Mother. 

In returning, you rest in the Source 
and are free from harm.

\section{The Eye of the Needle}
The world says, ``Strive to enter the narrow gate.'' 
But I tell you: the gate is narrow 
only because you carry the burden of the many. 

It is easier for a camel 
to pass through the eye of a needle 
than for one rich in self 
to enter the Kingdom. 

Become small, become single, 
and the narrow path widens into sky.

\section{The Unshakable Root}
What is rooted in the Source 
cannot be uprooted. 
What is founded in the Light 
cannot be shaken. 

Cultivate this root in yourself, 
and harmony spreads around you 
without a word.

\ornament

% =========================================================
% BOOK 7
% =========================================================
\chapter{The Unforced Life}
\begin{center}
    	extit{(Sayings 55--63)} 
    \vspace{0.5cm}
    \small Do not force the river. \ Be the water.
\end{center}
\vspace{0.5cm}

\section{The Child of the Light}
One who lives in the Light 
is like a newborn child— 
unafraid of serpents, 
unmoved by thunder, 
unharmed by illusions. 

The bones are soft, 
yet the strength is great.

\section{The One Who Knows}
The Spirit speaks most clearly 
when the tongue is still. 

The empty vessel echoes loudly; 
the full vessel is quiet. 

Yet I speak through silence 
more plainly than all the world's voices. 

Seal the senses, quiet the mind, be still— 
and you behold the All.

\section{Non-Interference}
The more laws you make, 
the more thieves arise. 
The more weapons you forge, 
the more fear you sow. 

Transform the world 
by transforming yourself. 
Then the world changes 
of its own accord.

\section{The Mystery of Opposites}
When rulers are gentle, 
people grow clear. 
When rulers are harsh, 
people grow cunning. 

Misfortune hides inside good fortune; 
good fortune hides inside misfortune. 

The wise trust neither shadow, 
but cling to the Real.

\section{The Economy of Spirit}
Restrain desire, conserve your breath, return to the root. 

Those who spend their strength 
inwardly 
become inexhaustible.

\section{Governing by Non-Grasping}
Leaven the soul as you would the dough: 
gently, until the whole loaf rises. 

The Single One rules by not ruling, 
teaches by not preaching, 
achieves by not striving. 

Thus the world settles 
in such a presence.

\section{The Great Acceptance}
The Kingdom is like the Great Sea. 
It refuses no river, 
judges no stream. 

It receives the muddy and the clear alike, 
and in its vastness 
all become one.

\section{The Treasure Within}
The Way is the refuge of all beings— 
treasure of the good, 
sanctuary of the not-good. 

Beautiful words can inspire, 
but a silent heart 
transforms.

\section{The Small Actions}
Do the great work through small acts. 
Untie the knot while it is loose. 
Tend the vine while it is young. 

Answer hatred with peace. 

The one who refuses to harm 
cannot be harmed.

\ornament

% =========================================================
% BOOK 8
% =========================================================
\chapter{The Wisdom of the Child}
\begin{center}
    	extit{(Sayings 64--72)} 
    \vspace{0.5cm}
    \small The wise are not learned. \ The learned are not wise.
\end{center}
\vspace{0.5cm}

\section{The First Step}
A great tree grows from a tiny seed. 
A tower of nine stories 
begins with a single heap of earth. 

The journey to the Father 
begins where your feet stand now.

\section{Innocence as Wisdom}
The ancient sages did not try 
to make people clever; 
they helped them return 
to simplicity. 

The more tricks people learn, 
the further they drift from the Way.

\section{Leading by Following}
Rivers rule the land 
because they flow below it. 

Thus the Single One leads 
by remaining beneath. 

Because there is no competition, 
no one contends.

\section{The Three Jewels}
I hold three treasures: 
gentleness, 
simplicity, 
and not placing myself above anyone. 

Keep these, 
and you will walk in the Light.

\section{The Peaceful Warrior}
The greatest warrior does not fight. 
The greatest general does not stir anger. 
The greatest victory leaves no wounds. 

Master yourself, 
and the world is mastered.

\section{The Paradox of Yielding}
There is no greater misfortune 
than making war on your own soul. 

Underestimate this inner opponent, 
and you are already defeated. 

Yield, 
and the false self collapses.

\section{A Teaching Few Understand}
My words are simple to hear, 
hard to embody. 

Many read the teachings; 
few wear them as skin. 

The one who truly embodies the Way 
walks unseen.

\section{The Gift of Not-Knowing}
To know that you do not know— 
this is clarity. 

To think you know 
while moving in ignorance— 
this is blindness. 

The awakened remove the veil 
from their own eyes first.

\section{The Fear of the False Self}
When people fear only opinions, 
their hearts grow small. 

Return to your true nature, 
and fear dissolves— 
for what can threaten 
the one who knows the self is eternal?

\ornament

% =========================================================
% BOOK 9
% =========================================================
\chapter{The Return to Source}
\begin{center}
    	extit{(Sayings 73--81)} 
    \vspace{0.5cm}
    \small The Way of Heaven is to benefit and not to harm. \ The Way of the Sage is to act but not to compete.
\end{center}
\vspace{0.5cm}

\section{The Courage of the Way}
Daring without wisdom 
leads to death. 
Wisdom without daring 
leads nowhere. 

The Way grants a courage that does not wound, 
and a power that does not dominate.

\section{The One Who Judges}
Why fear death? 
Only the Living Source 
dissolves forms. 

Those who pretend to judge life and death 
are like children 
playing at steering a great chariot— 
dangerous to themselves and others.

\section{The Burden of Excess}
People suffer 
because they cling to excess. 

The Single One lives lightly, 
needing little, 
carrying no burden, 
fearing no loss.

\section{The Green Wood and the Dry}
Green wood is full of sap and bends. 
Dry wood is brittle and snaps. 

What yields belongs to the Living One; 
what resists belongs to the grave. 

The one who remains supple 
cannot be broken.

\section{The Winnowing Fan}
The Father's way is like a winnowing fan: 
it lifts what is mixed, 
separates grain from chaff, 
gathers what nourishes, 
scatters what is empty. 

So too the Single One gives to the needy 
without claiming virtue.

\section{The Stone and the Corner}
Nothing is softer than water, 
yet nothing overcomes stone more completely. 

The stone the builders rejected 
becomes the cornerstone. 

So too the gentle, despised as weak, 
overcome the strong.

\section{The End of Debts}
Even after a truce, 
resentment lingers. 

But the awakened keep no accounts. 
They see no debtor, no creditor— 
only One Light 
in many forms.

\section{The Simple Kingdom}
Imagine a small, quiet land 
where people taste simplicity 
and lose their hunger for excess. 

Such a kingdom lives within you. 
Dwell there, 
and the outer world grows gentle.

\section{The Completion}
Truth does not wear fine clothes. 
Fine clothes often hide a lie. 

I give freely, not possessing. 
The more the Single One gives, 
the greater the store— 
for giving flows 
from the inexhaustible Source. 

Thus the scroll ends where the Way begins: 
in Silence, in Seeing, 
in the Light that you are.

\ornament
\section{The Inner and the Outer}
The cup is useful for its emptiness. \\
So too the self is useful \\
when it becomes empty of self.

Inside and outside are mirrors. \\
When they reflect one another without distortion, \\
the All is revealed.

\section{The Lion and the Child}
The world teaches you to be a lion— \\
to conquer, claim, and compel. \\
But I tell you: \\
unless the lion becomes a child, \\
you cannot know the Living Source.

The child is empty, open, and whole. \\
There is no image of self to defend. \\
This is why the child enters the Kingdom easily.

\section{The Unbinding}
What binds you is not outside you. \\
What frees you is not outside you. \\
The prison and the key \\
are fashioned from the same ignorance.

Awaken, and you will laugh— \\
for there are no gates.

\section{The Return to the Unborn}
Before you were a form, you were the Living Breath. \\
Before you were breath, you were the Light. \\
Before you were the Light, \\
you were with the Source.

To return is not to go backward. \\
It is to remember the place \\
that has never moved.

\section{The Silence That Speaks}
Words are nets thrown into the sea. \\
They catch small fish, \\
but the Great Fish swims free.

Seek not many words from me. \\
Seek the silence in which the Word is born. \\
There you and I are one life, \\
one breath, \\
one unbroken Being.

\section{The Great Stillness}
Empty yourself of what you think you are. \\
Let the waters of the heart become clear. \\
In stillness, the sediment settles by itself, \\
and the bottom is revealed.

In the same way, \\
when the soul becomes still, \\
the Light of the Living Source \\
shines through without effort.

\section{The Hidden Guide}
The greatest guide is the one who disappears. \\
The seeker awakens and says: \\
``Look—I have found the Way myself.''

Thus I leave no trace, \\
and yet my footsteps are everywhere.

\section{When the Way Is Forgotten}
When the Way is forgotten, virtue is invented. \\
When virtue is lost, rules are created. \\
When rules fail, people cling to punishment and reward.

The further you descend, \\
the further you move from Life. \\
Return to the beginning, \\
and the rules fall away like old clothing.

\ornament

% =========================================================
% BOOK 3
% =========================================================
\chapter{The Garment of Silence}
\begin{center}
    \textit{(Sayings 19--27)} \\
    \vspace{0.5cm}
    \small The garment divides; the wearer unites.
\end{center}
\vspace{0.5cm}

\section{Beyond the Teachings}
Abandon holiness; return to wholeness. \\
Abandon righteousness; return to sight. \\
Abandon the law; return to the Source.

What you call ``good'' and ``evil'' \\
are shadows cast by your own dividing mind. \\
The heart that sees truly needs no commandments.

\section{Not of This World}
The world shouts; the Way whispers. \\
The world demands belief; the Way asks you to see. \\
The world praises power; the Way dissolves it.

Many chase desires. \\
Few seek liberation. \\
Yet the Kingdom belongs to those \\
who loosen their grip on everything.

\section{The Face of the Invisible}
The Way is a child-like mystery— \\
seen only by those who forget themselves.

It has no beginning and no end. \\
It moves without moving. \\
Its face cannot be drawn, \\
yet it shines through your own face \\
when you remember who you are.

\section{The One Who Is Bent}
If you wish to be whole, let yourself be broken. \\
If you wish to be full, empty yourself. \\
If you wish to be reborn, die to your false image.

I do not raise the proud; \\
I raise only the one \\
who has laid the self down.

\section{The Wind Speaks Briefly}
Speak only what is true, and your words will be few. \\
The wind does not shout; \\
yet entire forests bow before it.

When you speak from the Living Source, \\
your words carry no force— \\
yet they move mountains.

\section{The Tower That Topples}
One who raises the self has already fallen. \\
One who displays virtue has already lost it. \\
One who demands honor has already betrayed the soul.

Stand in your true nature— \\
and honor flows from you without asking.

\section{The Womb of Silence}
There is a womb older than creation. \\
It is the Silence that births the All. \\
It is formless, unbounded. \\
It moves through you as breath, \\
yet it is not the breath.

Those who know Her \\
do not cling to life nor fear death.

\section{The Heavy Root}
The root of the tree is unseen, \\
yet it holds the whole tree upright. \\
So too the silent depth of your being \\
supports your every step.

Those who forget their root \\
are blown about by every wind— \\
desire, fear, anger, ignorance. \\
Remember your depth, and nothing can uproot you.

\section{The Footprints of the Master}
A master walks without leaving marks. \\
Speaking without wounding. \\
Guiding without controlling.

To the blind—a lamp; \\
to the lost—a doorway; \\
to the weary—rest. \\
Seeing the divine spark in all beings, \\
none are beyond help.

\ornament

% =========================================================
% BOOK 4
% =========================================================
\chapter{The Power of the Gentle}
\begin{center}
    \textit{(Sayings 28--36)}
\end{center}
\vspace{0.5cm}

\section{The Fertile Soil}
Know the strength of the sower, \\
yet keep to the softness of the soil— \\
the fertile field that receives the seed.

Know knowledge, yet return to innocence. \\
Thus you become the place \\
where the two become one.

\section{The Futility of Control}
Those who attempt to seize the world will mar it. \\
The world is a sacred vessel; \\
it cannot be shaped by human hands.

Force cracks the clay. \\
Grasping spills the wine. \\
Only the open hand receives the fullness of life.

\section{The Way of Non-Violence}
Those who walk in the Way do not harm. \\
They do not conquer. \\
They do not take up the sword.

For every blow you strike strikes you in return. \\
Every wound you inflict \\
becomes your own garment of suffering.

Power gained through violence rots the soul; \\
power gained through awakening cannot be taken away.

\section{The Weapon of the Heart}
Weapons are tools of fear; \\
the heart of the awakened has no use for them. \\
Victory through harm is mourning in disguise.

The wise triumph by dissolving conflict, \\
not by defeating an opponent— \\
seeing no opponent at all.

\section{The Way, the Truth}
People ask, ``Show us the path.'' \\
I tell them: \textbf{I am the Way.} \\
People ask, ``Show us the doctrine.'' \\
I tell them: \textbf{I am the Truth.} \\
People ask, ``Show us immortality.'' \\
I tell them: \textbf{I am the Living One.}

The Way is not a map; it is the Single One walking. \\
No one comes to the Father of Light \\
except by becoming what I am.

\section{Profit and Loss}
To know others is intelligence. \\
To know the self is illumination. \\
To conquer others is strength. \\
To conquer the self is freedom.

For what profit is there in gaining the ten thousand things \\
if you lose the One who perceives them?

\section{The Great River}
The Way flows everywhere, \\
giving itself to all beings, claiming nothing.

One who walks in this Way acts without self, \\
gives without calculation, \\
and rests without pride. \\
Thus becoming as the Living Source itself.

\section{The Face of Peace}
Hold fast to the image of the Living One \\
and all beings come to rest in your presence.

Peace is not the absence of conflict; \\
it is the recognition that conflict was a dream. \\
The awakened do not persuade—they radiate.

\section{The Paradox of Power}
To let something expand, first allow it to contract. \\
To let something grow strong, first allow it to weaken.

Thus I guide by reversal, \\
teaching the soul through its own emptiness.

\ornament

% =========================================================
% BOOK 5
% =========================================================
\chapter{The Union of Opposites}
\begin{center}
    \textit{(Sayings 37--45)}
\end{center}
\vspace{0.5cm}

\section{The Ease of the Way}
The Way acts without effort. \\
When the heart aligns with it, desire loosens its grip.

But when desire rises, the soul becomes troubled. \\
Use the Light to calm the waters, \\
and they become clear again.

\section{True Virtue}
High virtue does not display itself; \\
it acts from the Source effortlessly. \\
Lesser virtue clings to rules \\
because it has forgotten its origin.

When the Kingdom is remembered, virtue blossoms naturally. \\
When it is forgotten, people invent commandments.

\section{The Ones Who Remain Whole}
Heaven remains Heaven because it does not exalt itself. \\
The earth remains earth because it does not resist its nature. \\
And the soul remains whole when it refuses to be divided.

\section{The Return}
The movement of the Way is return— \\
not to the past, but to the unborn Light within you. \\
All things rise from the Source, \\
and the Source rises from silence.

\section{The Three Seekers}
The wise hear the teaching and practice it deeply. \\
The average hear it and practice it partially. \\
The foolish hear it and laugh aloud.

Yet the Way dances in the laughter as well, \\
for all beings walk toward awakening, \\
even when they walk away.

\section{The Birth of the Two}
The Source gives birth to the Father and the Mother. \\
The Two give birth to the Many. \\
The Many return to the One.

He who sees the One in every form and motion \\
cannot be shaken.

\section{The Gentle Overcomes}
Truly I say to you: Water, soft and yielding, \\
cuts through the rock of ages. \\
What has no substance enters where nothing can enter. \\
Thus the soul, when subtle, passes through every barrier.

\section{Treasure and Self}
Fame or life—which is more precious? \\
Gain or the soul—which is more valuable?

The one who knows their true nature \\
cannot be seduced away from it.

\section{The Great Fulness}
The cup most full appears most empty. \\
The path most straight appears crooked. \\
The hand most skilled trembles.

Why? Because the world sees only surfaces,\hfill\\
and the Real dwells beneath the seen. \\

The awakened make peace with paradox \\
and so are not deceived by appearances.

\ornament

% =========================================================
% BOOK 6
% =========================================================
\chapter{The Empty Vessel}
\begin{center}
    \textit{(Sayings 46--54)}
\end{center}
\vspace{0.5cm}

\section{The Yoke of Peace}
When the Way is lived, the ox knows its master's stall. \\
When the Way is forgotten, the sword is drawn in the street.

The yoke of the world is heavy with desire. \\
My yoke is easy, and brings peace.

\section{Seeing Without Seeking}
Without leaving home, you may know the whole world. \\
Without looking through the eyes, you may see the Living Source.

The further you flee outward, \\
the further you stray inward.

\section{The Unlearning}
The scholar gathers more every day. \\
The seeker of the Way loses more every day. \\
She sheds the garments of opinion \\
until only the naked Truth remains.

He who lets go holds the All.

\section{The Heart of the Single One}
The Single One is good to those who are good, \\
and good to those who are not good— \\
seeing only the Light within them.

Trusting those who trust, \\
and trusting those who do not trust— \\
for trust flows from the inner being, not theirs.

\section{Life and Death}
Those who cling to life fear death. \\
Those who know Me fear neither.

Death is but the undressing of the bride \\
before she enters the chamber.

\section{The Nourishing Way}
The Way gives birth, nourishes, guides, protects, \\
and does so without claiming ownership.

Thus the Single One completes the work and forgets the work— \\
and it remains forever.

\section{Return to the Mother}
Know the Mother, and you know the children. \\
Know the children, and return to the Mother. \\
In returning, you dwell in the Source and are free from harm.

\section{The Eye of the Needle}
The world says, ``Strive to enter the narrow gate.'' \\
But I tell you: The gate is narrow only because you carry the burden of the many.

It is easier for a camel to pass through the eye of a needle \\
than for the one rich in self to enter the Kingdom. \\
Become small, become single, \\
and the narrow path becomes wide as the sky.

\section{The Unshakable Root}
What is rooted in the Source cannot be uprooted. \\
What is established in the Light cannot be shaken.

Cultivate this root in yourself, \\
and harmony spreads to all around you \\
without your speaking a word.

\ornament

% =========================================================
% BOOK 7
% =========================================================
\chapter{The Unforced Life}
\begin{center}
    \textit{(Sayings 55--63)}
\end{center}
\vspace{0.5cm}

\section{The Child of the Light}
One who lives in the Light is like a newborn child— \\
unafraid of serpents, unmoved by loud voices, unharmed by illusions. \\
The bones are soft, yet the strength is great.

\section{The One Who Knows}
The Spirit speaks only when the tongue is still. \\
The empty vessel echoes loudly; the full vessel is silent. \\
Yet I speak through silence \\
more clearly than all the world's voices.

Seal the senses. Quiet the mind. \\
Be still—and you behold the All.

\section{Non-Interference}
The more laws you create, the more thieves appear. \\
The more weapons you forge, the more fear grows.

Transform the world by transforming yourself. \\
Then the world transforms naturally.

\section{The Mystery of Opposites}
When rulers are gentle, people thrive. \\
When rulers are harsh, people grow cunning.

Misfortune hides in good fortune; \\
good fortune hides in misfortune. \\
The wise trust neither shadow, but cling only to the Real.

\section{The Economy of Spirit}
Restrain desire. Conserve energy. Return to the root. \\
Those who spend life strengthening the inner being \\
become inexhaustible.

\section{Governing by Non-Grasping}
Guide the soul as you would leaven the dough: \\
gently, until the whole loaf rises. \\
The Single One governs by not ruling, \\
teaches by not preaching, \\
achieves by not striving. \\
Thus the world settles in such presence.

\section{The Great Acceptance}
The Kingdom is like the Great Sea. \\
It refuses no river. \\
It judges no stream. \\
It receives the muddy and the clear alike, \\
and in its vastness, all become one.

\section{The Treasure Within}
The Way is the refuge of all beings, \\
the treasure of the good and the sanctuary of the not-good.

Beautiful words can inspire, \\
but a silent heart transforms.
\section{The Small Actions}
Do the great work through small acts. \\
Respond to hatred with peace. \\
Untie the knot while it is loose. \\
Tend the vine while it is young.

The one who refuses to harm cannot be harmed.

\ornament

% =========================================================
% BOOK 8
% =========================================================
\chapter{The Wisdom of the Child}
\begin{center}
    \textit{(Sayings 64--72)}
\end{center}
\vspace{0.5cm}

\section{The First Step}
A great tree grows from a tiny seed. \\
A tower nine stories high begins with a heap of earth. \\
The journey to the Father \\
begins exactly where your feet stand now.

\section{Innocence as Wisdom}
The ancient sages did not attempt to enlighten the people— \\
they helped them return to simplicity. \\
The more cleverness people acquire, \\
the further they drift from the Way.

\section{Leading by Following}
Rivers rule the land because they flow beneath it. \\
Thus the Single One rules by staying below. \\
Because of non-competition, \\
no one competes.

\section{The Three Jewels}
I have three treasures: \\
gentleness, simplicity, and not putting myself above anyone. \\
Keep these, and you will walk in the Light.

\section{The Peaceful Warrior}
The greatest warrior does not fight. \\
The greatest general does not stir anger. \\
The greatest victory leaves no wounds.

Master yourself, and the world is mastered.

\section{The Paradox of Yielding}
There is no greater misfortune than underestimating your opponent— \\
your opponent being your own deluded self. \\
Yield, and the false self collapses.

\section{A Teaching Few Understand}
My words are easy to hear but difficult to realize. \\
Many read the teachings, but few wear them as skin. \\
The Single One who embodies the Way walks unseen.

\section{The Gift of Not-Knowing}
To know that you do not know—this is clarity. \\
To think you know while living in ignorance—this is blindness. \\
The awakened remove the cataract from their own sight first.

\section{The Fear of the False Self}
When people fear only the opinions of others, their hearts shrink. \\
Return to your true nature, and fear dissolves— \\
for what can threaten the one who knows the self is eternal?

\ornament

% =========================================================
% BOOK 9
% =========================================================
\chapter{The Return to Source}
\begin{center}
    \textit{(Sayings 73--81)}
\end{center}
\vspace{0.5cm}

\section{The Courage of the Way}
Daring without wisdom leads to death. \\
Wisdom without daring leads nowhere. \\
The Way gives courage that does not wound \\
and power that does not dominate.

\section{The One Who Judges}
Why fear death? \\
The Living Source alone dissolves forms. \\
Those who attempt to take the place of the eternal \\
are like a child pretending to steer a great chariot— \\
dangerous to self and others.

\section{The Burden of Excess}
People suffer because they cling to excess. \\
The Single One lives lightly, needing little. \\
Carrying no burden, fearing no loss.
\section{The Green Wood and the Dry}
When the wood is green, it is full of sap and bends. \\
When the wood is dry, it is brittle and snaps. \\
That which yields belongs to the Living One; \\
that which resists belongs to the grave. \\
The one who remains supple cannot be broken.

\section{The Winnowing Fan}
The Father's way is like the winnowing fan: \\
separating the grain from the chaff, \\
gathering the worthy and scattering the empty. \\
So too the Single One gives to the needy without claiming virtue.

\section{The Stone and the Corner}
Nothing is softer than water, \\
yet nothing overcomes stone more completely. \\
The stone that the builders rejected \\
has become the cornerstone of the temple. \\
So too the gentle overcomes the strong.

\section{The End of Debts}
Even after a truce, resentment lingers. \\
But the awakened hold no accounts. \\
They see no debtor, no creditor— \\
only the One Light in many forms.

\section{The Simple Kingdom}
Imagine a small, peaceful land \\
where people taste simplicity \\
and lose the appetite for excess. \\
Such a kingdom is within you. \\
Live there, and the world outside becomes gentle.

\section{The Completion}
Truth does not wear fine clothes. \\
Fine clothes often hide a lie. \\
I give freely, without possessing.
The more the Single One gives, the greater the store— \\
for giving comes from the inexhaustible Source.

Thus the scripture ends where the Way begins: \\
in silence, in seeing, in the Light that you are.

\ornament

% --- BACK MATTER ---
\chapter{Glossary of the Inner Kingdom}
\small
\textbf{Gnosis:} Knowledge that is not intellectual or doctrinal, but direct, experiential knowing of the Divine. It is the recognition of one's own divine origin.

\textbf{The Single One (Monachos):} A term from the Gospel of Thomas for the solitary or unified seeker; one who has integrated the inner opposites and become whole.

\textbf{Nous:} The "Eye of the Heart" or spiritual intellect. It is the faculty within the human soul that perceives the Divine directly, distinct from the rational mind.

\textbf{The Bridal Chamber:} A Gnostic metaphor for the state of union where duality (male/female, human/divine, inner/outer) is dissolved and the soul is reunited with the Spirit.

\textbf{The Mother:} The feminine aspect of the Divine (often associated with the Holy Spirit in early Semitic Christian texts), representing Wisdom (Sophia), silence, and the womb of creation.

\textbf{The Father:} The masculine aspect of the Divine, representing the originating Light, the unmanifest structure, and the Will. In this scroll, Father and Mother are the two hands of the One Source.

\textbf{The Way (Tao):} The flow of the Universe; the unnameable Source that orders all things without force. In this scroll, it is identified with the Kingdom of Heaven spread upon the earth.

\chapter{Colophon}

\vspace{2cm}

\begin{center}
    \textit{This Scroll was copied in the stillness of the turning year, \\
    by a hand seeking only to disappear, \\
    that the Light might remain.}
    
    \vspace{1cm}
    
    $\odot$
    
    \vspace{1cm}
    
    \textbf{May Peace Be Upon the Reader.}
\end{center}

\vspace{3cm}

\begin{center}
\textbf{End of Manuscript}
\end{center}

\end{document}